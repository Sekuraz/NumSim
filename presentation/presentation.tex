\documentclass[aspectratio=169]{beamer}
\mode<presentation>

\usepackage[T1]{fontenc}
\usepackage[utf8]{inputenc}
\usepackage{lmodern}
\usepackage[ngerman]{babel}
\usepackage{presentationST}
\usepackage{hyperref}
\usepackage{pgfplots}\pgfplotsset{compat=1.13} % necessary?
\usepackage{multimedia}
\usepackage{multimedia}

\hypersetup{unicode=true, pdftoolbar=true, pdfmenubar=true, pdffitwindow=false, 
	pdfstartview={FitH},
	pdftitle={Numerische Simulation: Mehrgitter und Konjugierte Gradienten Methode},
	pdfauthor={Stephan Lunowa, Jonas Harsch, Markus Baur},
	pdfsubject={Mehrgitter, Konjugierte Gradienten, Wintersemester 2016/17},
	pdfcreator={\LaTeX\ with ptextbfkage \flqq hyperref\frqq},
	pdfproducer={pdfTeX \the\pdftexversion.\pdftexrevision},
	pdfnewwindow=true,
	colorlinks=true,linkcolor=black,citecolor=black,filecolor=magenta,urlcolor=black}

%%%%%%%%%%%%%%%%%%%%%%%%%%%%%%%%%%%%%%%%%%%%%%%%%%%%%%%%%%%%%%%%%%%%%%%%%%%%
% For Printing
%%%%%%%%%%%%%%%%%%%%%%%%%%%%%%%%%%%%%%%%%%%%%%%%%%%%%%%%%%%%%%%%%%%%%%%%%%%%
%\usepackage{pgfpages}
%\pgfpagesuselayout{resize to}[a4paper,border shrink=5mm,landscape]
%Zum Drucken von 8 Folien pro Seite setzt man die Layoutoptionen entsprechend des ntextbfhfolgenden Beispiels.
%\pgfpagesuselayout{8 on 1}[a4paper,border shrink=5mm]

%%%%%%%%%%%%%%%%%%%%%%%%%%%%%%%%%%%%%%%%%%%%%%%%%%%%%%%%%%%%%%%%%%%%%%%%%%%%
% Mathematical Symbols and blocks
%%%%%%%%%%%%%%%%%%%%%%%%%%%%%%%%%%%%%%%%%%%%%%%%%%%%%%%%%%%%%%%%%%%%%%%%%%%%
\newcommand{\N}{\mathbb{N}}
\newcommand{\R}{\mathbb{R}}
\newcommand{\diff}[1]{\partial_{#1}}
\DeclareMathOperator{\divergence}{div}
\newtheoremstyle{thm}%
  {}%         Sptextbfe above, empty = `usual value'
  {}%         Sptextbfe below
  {\itshape}% Body font
  {}%         Indent amount (empty = no indent, \parindent = para indent)
  {\bfseries}% Thm head font
  {}%         Punctuation after thm head
  {\newline}% Sptextbfe after thm head: \newline = linebreak
  {}%         Thm head spec
\theoremstyle{thm}
%\newtheorem{theorem}{Satz}[section] % schon definiert
%\newtheorem{lemma}[theorem]{Lemma} % schon definiert
%\newtheorem{definition}[theorem]{Definition} % schon definiert
\newtheorem{assumption}[theorem]{Annahme}
\newtheorem{remark}[theorem]{Bemerkung}

%%%%%%%%%%%%%%%%%%%%%%%%%%%%%%%%%%%%%%%%%%%%%%%%%%%%%%%%%%%%%%%%%%%%%%%%%%%%
% Titlepage
%%%%%%%%%%%%%%%%%%%%%%%%%%%%%%%%%%%%%%%%%%%%%%%%%%%%%%%%%%%%%%%%%%%%%%%%%%%%
\title{Mehrgitter und Konjugierte Gradienten Methode}
\subtitle{Numerische Simulation}
\author{Markus Baur \and Jonas Harsch \and Stephan Lunowa}
\institute[IPVS]{Prof. Dr. rer. nat. habil. M. Mehl\\
Institut für Parallele und Verteilte Systeme\\
Universität Stuttgart}
\date{\today}

%%%%%%%%%%%%%%%%%%%%%%%%%%%%%%%%%%%%%%%%%%%%%%%%%%%%%%%%%%%%%%%%%%%%%%%%%%%%
% Document
%%%%%%%%%%%%%%%%%%%%%%%%%%%%%%%%%%%%%%%%%%%%%%%%%%%%%%%%%%%%%%%%%%%%%%%%%%%%
\begin{document}

\frame[plain]{\mbox{}\vspace{2em}\titlepage}
\frame[plain]{\frametitle{Inhalt} \setcounter{tocdepth}{1} \tableofcontents }
% framenumber neu setzen, damit Nummerierung bei 1 anfängt.
%\setcounter{framenumber}{0}

%%%%%%%%%%%%%%%%%%%%%%%%%%%%%%%%%%%%%%%%%%%%%%%%%%%%%%%%%%%%%%%%%%%%%%%%%%%%
% Mehrgitter Methode
%%%%%%%%%%%%%%%%%%%%%%%%%%%%%%%%%%%%%%%%%%%%%%%%%%%%%%%%%%%%%%%%%%%%%%%%%%%%
\section{Mehrgitter Methode}\label{sec:MG}
\begin{frame}{Mehrgitter Methode (MG)}
  \begin{itemize}[<+(1)->]
    \item Multigrid / Multilevel / Mehrgitter Methode
    \item \textcolor{red}{\huge TODO}
  \end{itemize}
\end{frame}

%%%%%%%%%%%%%%%%%%%%%%%%%%%%%%%%%%%%%%%%%%%%%%%%%%%%%%%%%%%%%%%%%%%%%%%%%%%%
% Konjugierte Gradienten Methode
%%%%%%%%%%%%%%%%%%%%%%%%%%%%%%%%%%%%%%%%%%%%%%%%%%%%%%%%%%%%%%%%%%%%%%%%%%%%
\section{Konjugierte Gradienten Methode}\label{sec:CG}
\begin{frame}{Konjugierte Gradienten Methode (CG)}
  \begin{itemize}[<+(1)->]
    \item \textcolor{red}{\huge TODO}
  \end{itemize}
\end{frame}

%%%%%%%%%%%%%%%%%%%%%%%%%%%%%%%%%%%%%%%%%%%%%%%%%%%%%%%%%%%%%%%%%%%%%%%%%%%%
% Numerische Ergebnisse
%%%%%%%%%%%%%%%%%%%%%%%%%%%%%%%%%%%%%%%%%%%%%%%%%%%%%%%%%%%%%%%%%%%%%%%%%%%%
\section{Numerische Ergebnisse}\label{sec:Ergebnisse}
\begin{frame}{Numerische Ergebnisse}
  \centering \textcolor{red}{\huge TODO}
  \begin{tikzpicture}\begin{loglogaxis}[width=8cm,
    xlabel={\#Gitterpunkte $N$ in einer Dimension}, ylabel={\#Iterationen}, grid=minor,
    log basis x=2, legend pos= outer north east,
    legend entries={MG, CG, RBSOR, SOR, $\mathcal{O}(N)$},
    domain=4:128]
    \addplot table {it_MG.csv};
    \addplot table {it_CG.csv};
    \addplot table {it_RB.csv};
    \addplot table {it_SOR.csv};
    \addplot[black, dashed] {8*x};
  \end{loglogaxis}\end{tikzpicture}
\end{frame}

\begin{frame}{Numerische Ergebnisse}
  \centering
  \begin{tikzpicture}\begin{loglogaxis}[width=8cm,
    xlabel={\#Gitterpunkte $N$ in einer Dimension}, ylabel={Laufzeit}, grid=minor,
    log basis x=2, legend pos= outer north east,
    legend entries={MG, CG, RBSOR, SOR, $\mathcal{O}(N^3)$, $\mathcal{O}(N^2)$},
    domain=8:128]
    \addplot table {sequential/drivencavity_MG.csv};
    \addplot table {sequential/drivencavity_CG.csv};
    \addplot table {sequential/drivencavity_RBSOR.csv};
    \addplot table {sequential/drivencavity_SOR.csv};
    \addplot[black, dashed] {x^3/2000};
    \addplot[black, dotted] {x^2/600};
  \end{loglogaxis}\end{tikzpicture}
\end{frame}

\begin{frame}{Numerische Ergebnisse}
  \centering
  \begin{tikzpicture}\begin{loglogaxis}[width=8cm,
    xlabel={\#Gitterpunkte $N$ in einer Dimension}, ylabel={Laufzeit}, grid=minor,
    log basis x=2, legend pos= outer north east,
    legend entries={MG, CG, RBSOR, SOR, $\mathcal{O}(N^3)$, $\mathcal{O}(N^2)$},
    domain=8:128]
    \addplot table {parallel/drivencavity_MG.csv};
    \addplot table {parallel/drivencavity_CG.csv};
    \addplot table {parallel/drivencavity_RBSOR.csv};
    \addplot table {parallel/drivencavity_SOR.csv};
    \addplot[black, dashed] {x^3/2000};
    \addplot[black, dotted] {x^2/600};
  \end{loglogaxis}\end{tikzpicture}
\end{frame}

%%%%%%%%%%%%%%%%%%%%%%%%%%%%%%%%%%%%%%%%%%%%%%%%%%%%%%%%%%%%%%%%%%%%%%%%%%%%
% Fazit & Ausblick
%%%%%%%%%%%%%%%%%%%%%%%%%%%%%%%%%%%%%%%%%%%%%%%%%%%%%%%%%%%%%%%%%%%%%%%%%%%%
\section{Fazit und Ausblick}\label{sec:Fazit}
\begin{frame}{Fazit und Ausblick}
  \begin{itemize}[<+(1)->]
    \item \textcolor{red}{\huge TODO}
    \item Ausblick:
      \begin{itemize}
        \item Mehrgitter für Gitter mit Hindernissen
        \item Full Multigrid und Full Approximation Scheme
        \item Vorkonditionierte CG Verfahren
        \item Approximation mit Operatoren höherer Ordnung (Rand und im Gebiet)
        \item Kombination CG und MG
      \end{itemize}
  \end{itemize}
\end{frame}

\begin{frame}[plain]
  \begin{center}
    \Large \textcolor{simtechred}{ Vielen Dank für die Aufmerksamkeit } \\[2em]
    \normalsize Noch Fragen?
  \end{center}
\end{frame}

\end{document}
