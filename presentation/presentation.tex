\documentclass[aspectratio=169]{beamer}
\mode<presentation>

\usepackage[T1]{fontenc}
\usepackage[utf8]{inputenc}
\usepackage{lmodern}
\usepackage[ngerman]{babel}
\usepackage{presentationST}
\usepackage{hyperref}
\usepackage{pgfplots}\pgfplotsset{compat=1.13} % necessary?
\usepackage{multimedia}
\usepackage{algpseudocode}
\usepackage{tikz}
\usetikzlibrary{patterns}
\algnewcommand{\IIf}[1]{\State\algorithmicif\ #1\ \algorithmicthen}
\algnewcommand{\EndIIf}{\unskip\ \algorithmicend\ \algorithmicif}
\renewcommand{\Comment}[2][.5\linewidth]{\leavevmode\hfill\makebox[#1][l]{//~#2}}

\hypersetup{unicode=true, pdftoolbar=true, pdfmenubar=true, pdffitwindow=false, 
	pdfstartview={FitH},
	pdftitle={Numerische Simulation: Mehrgitter und Konjugierte Gradienten Methode},
	pdfauthor={Stephan Lunowa, Jonas Harsch, Markus Baur},
	pdfsubject={Mehrgitter, Konjugierte Gradienten, Wintersemester 2016/17},
	pdfcreator={\LaTeX\ with ptextbfkage \flqq hyperref\frqq},
	pdfproducer={pdfTeX \the\pdftexversion.\pdftexrevision},
	pdfnewwindow=true,
	colorlinks=true,linkcolor=black,citecolor=black,filecolor=magenta,urlcolor=black}

%%%%%%%%%%%%%%%%%%%%%%%%%%%%%%%%%%%%%%%%%%%%%%%%%%%%%%%%%%%%%%%%%%%%%%%%%%%%
% For Printing
%%%%%%%%%%%%%%%%%%%%%%%%%%%%%%%%%%%%%%%%%%%%%%%%%%%%%%%%%%%%%%%%%%%%%%%%%%%%
%\usepackage{pgfpages}
%\pgfpagesuselayout{resize to}[a4paper,border shrink=5mm,landscape]
%Zum Drucken von 8 Folien pro Seite setzt man die Layoutoptionen entsprechend des ntextbfhfolgenden Beispiels.
%\pgfpagesuselayout{8 on 1}[a4paper,border shrink=5mm]

%%%%%%%%%%%%%%%%%%%%%%%%%%%%%%%%%%%%%%%%%%%%%%%%%%%%%%%%%%%%%%%%%%%%%%%%%%%%
% Mathematical Symbols and blocks
%%%%%%%%%%%%%%%%%%%%%%%%%%%%%%%%%%%%%%%%%%%%%%%%%%%%%%%%%%%%%%%%%%%%%%%%%%%%
\newcommand{\N}{\mathbb{N}}
\newcommand{\R}{\mathbb{R}}
\newcommand{\diff}[1]{\partial_{#1}}
\DeclareMathOperator{\divergence}{div}
\newtheoremstyle{thm}%
  {}%         Sptextbfe above, empty = `usual value'
  {}%         Sptextbfe below
  {\itshape}% Body font
  {}%         Indent amount (empty = no indent, \parindent = para indent)
  {\bfseries}% Thm head font
  {}%         Punctuation after thm head
  {\newline}% Sptextbfe after thm head: \newline = linebreak
  {}%         Thm head spec
\theoremstyle{thm}
%\newtheorem{theorem}{Satz}[section] % schon definiert
%\newtheorem{lemma}[theorem]{Lemma} % schon definiert
%\newtheorem{definition}[theorem]{Definition} % schon definiert
\newtheorem{assumption}[theorem]{Annahme}
\newtheorem{remark}[theorem]{Bemerkung}

%%%%%%%%%%%%%%%%%%%%%%%%%%%%%%%%%%%%%%%%%%%%%%%%%%%%%%%%%%%%%%%%%%%%%%%%%%%%
% Titlepage
%%%%%%%%%%%%%%%%%%%%%%%%%%%%%%%%%%%%%%%%%%%%%%%%%%%%%%%%%%%%%%%%%%%%%%%%%%%%
\title{Mehrgitter und Konjugierte Gradienten Methode}
\subtitle{Numerische Simulation}
\author{Markus Baur \and Jonas Harsch \and Stephan Lunowa}
\institute[IPVS]{Prof. Dr. rer. nat. habil. M. Mehl\\
Institut für Parallele und Verteilte Systeme\\
Universität Stuttgart}
\date{\today}

%%%%%%%%%%%%%%%%%%%%%%%%%%%%%%%%%%%%%%%%%%%%%%%%%%%%%%%%%%%%%%%%%%%%%%%%%%%%
% Document
%%%%%%%%%%%%%%%%%%%%%%%%%%%%%%%%%%%%%%%%%%%%%%%%%%%%%%%%%%%%%%%%%%%%%%%%%%%%
\begin{document}

\frame[plain]{\mbox{}\vspace{2em}\titlepage}
\frame[plain]{\frametitle{Inhalt} \setcounter{tocdepth}{1} \tableofcontents }
% framenumber neu setzen, damit Nummerierung bei 1 anfängt.
%\setcounter{framenumber}{0}

%%%%%%%%%%%%%%%%%%%%%%%%%%%%%%%%%%%%%%%%%%%%%%%%%%%%%%%%%%%%%%%%%%%%%%%%%%%%
% Mehrgitter Methode
%%%%%%%%%%%%%%%%%%%%%%%%%%%%%%%%%%%%%%%%%%%%%%%%%%%%%%%%%%%%%%%%%%%%%%%%%%%%
\section{Mehrgitter Methode}\label{sec:MG}
\begin{frame}{Mehrgitter Methode (MG)}
  \begin{itemize}[<+(1)->]
    \item Multigrid / Multilevel / Mehrgitter Methode
    \item Iterative Löser schlecht für glatte Fehler
    \item Glatte Fehler auf gröberen Gittern oszillierend
    \item Mehrgitter hat $h$-unabhängige Konvergenzrate
    \item Essentiell: Beachtung der Randbedingungen um kleine Konvergenzfaktoren
        zu erhalten
    \item \textcolor{red}{\huge TODO}
  \end{itemize}
\end{frame}

\begin{frame}{Mehrgitter Methode (MG)}
  \begin{minipage}{0.49\linewidth}
  \begin{align*}
    \intertext{Im Gebiet:}
    \Delta p &= rhs + res \\
    \Delta e^c &= res^c = R_\Omega\ res
    \intertext{Analog auf dem Rand für Neumann-RB:}
    \partial_n p &= rhs + res \\
    \partial_n e^c &= res^c = R_\Gamma\ res
  \end{align*}
  \end{minipage}
  \begin{minipage}{0.49\linewidth}
  Beispiel linker Rand:\\
  \begin{tikzpicture}[scale=1.0]
    \draw [step=1.0cm, pattern=north east lines, opacity=0.5] (1.0,0.0) rectangle (2.0,1.0);
    \draw [step=1.0cm, pattern=north east lines, opacity=0.5] (1.0,1.0) rectangle (2.0,2.0);
    \draw [white, step=1.0cm, pattern=north east lines, opacity=0.5] (1.0,2.0) rectangle (2.0,2.5);
    \draw [white, step=1.0cm, pattern=north east lines, opacity=0.5] (1.0,0.0) rectangle (2.0,-0.5);
    \draw[step=2.0, red, ultra thick] (0.0, -0.5) grid (4.5, 2.5);
    \draw[step=1.0, black] (0.0, -0.5) grid (4.5, 2.5);

	\foreach \i in {1.5,...,3.5} {
      \foreach \j in {0.5,...,1.5} {
        \draw[blue, fill] (\i, \j) circle[radius=0.08cm];
      }
    }

    \draw[blue] (1.5, 0.5) node[above] {\scriptsize $p_{0,j}$};
    \draw[blue] (1.5, 1.5) node[above] {\scriptsize $p_{0,j+1}$};
    \draw[blue] (2.5, 0.5) node[above] {\scriptsize $p_{1,j}$};
    \draw[blue] (2.5, 1.5) node[above] {\scriptsize $p_{1,j+1}$};

	\foreach \i in {1.0,3.0} {
      \draw[red, fill] (\i, 1.0) circle[radius=0.08cm];
    }

	\draw[red] (1.0, 1.0) node[above, xshift=-0.25cm, yshift=-0.05cm] {\scriptsize $p^c_{0,j}$};
    \draw[red] (3.0, 1.0) node[above, xshift=-0.25cm, yshift=-0.05cm] {\scriptsize $p^c_{1,j}$};

    
    \draw[red, ultra thick] (4.8, 1.5) -- (5.2,1.5) node[right] {\textcolor{black}{grobes Gitter}};
    \draw[black] (4.8, 0.5) -- (5.2,0.5) node[right] {feines Gitter};
  \end{tikzpicture}
  \begin{align*}
    \tfrac{p_{1,j}-p_{0,j}}{h} &= rhs_{0,j} + res_{0,j} \\
    \tfrac{e^c_{1,j}-e^c_{0,j}}{2h} &= rhs^c_{0,j} = \tfrac{1}{2}(res_{0,j+1}+res_{0,j})
  \end{align*}
  \end{minipage}
\end{frame}

%%%%%%%%%%%%%%%%%%%%%%%%%%%%%%%%%%%%%%%%%%%%%%%%%%%%%%%%%%%%%%%%%%%%%%%%%%%%
% Konjugierte Gradienten Methode
%%%%%%%%%%%%%%%%%%%%%%%%%%%%%%%%%%%%%%%%%%%%%%%%%%%%%%%%%%%%%%%%%%%%%%%%%%%%
\section{Konjugierte Gradienten Methode}\label{sec:CG}
\begin{frame}{Konjugierte Gradienten Methode (CG)}
  \begin{itemize}[<+(1)->]
    \item Minimierung von $E(x) = \tfrac{1}{2}\langle Ax, x\rangle - \langle b,x \rangle$\\
        $\qquad\Rightarrow E'(x) = Ax - b \stackrel{!}{=} 0 \qquad\text{für}\quad A$
        symmetrisch positiv definit
    \item Suchrichtung in Gradientenrichtung liefert schnelle Konvergenz
    \item Ausschöpfen des kompletten $\R^N$  durch $A$-Konjugieren
        $\langle Ad_k, d_j\rangle = 0 \ \forall k \neq j$
    \item Optimierung des Algorithmus liefert als wesentlichen Aufwand eine
        Matrix-Vektor-Multiplikation
  \end{itemize}
\end{frame}

\begin{frame}[fragile]{Konjugierte Gradienten Methode (CG)}
  \small
  \begin{algorithmic}
    \Function{CG}{$A,b,x_0$} \Comment{Löst das LGS $Ax = b$}
    \State $r_0 := Ax - b$ \Comment{Initiales Residuum}
    \State $d_0 := r_0$    \Comment{Initiale Suchrichtung}
    \For{$k = 0, 1, \dots$}
      \State $z := Ad_k$
      \State $\alpha_k := \frac{\langle r_k, r_k \rangle}{\langle d_k, z\rangle}$
        \Comment{Optimale Suchweite}
      \State $x_{k+1} := x_k + \alpha_k d_k$
        \Comment{Korrektur der Lösung}
      \State $r_{k+1} := r_k - \alpha_k z$
        \Comment{Korrektur des Residuums}
      \State $\beta_k := \frac{\langle r_{k+1}, r_{k+1} \rangle}{\langle r_k, r_k\rangle}$
        \Comment{$A$-Konjugiertheit}
      \State $d_{k+1} := r_{k+1} + \beta_k d_k$
        \Comment{Korrektur der Suchrichtung}
      \State // Update der Randbedingungen
      \IIf{$\|r_{k+1}\| < \varepsilon$} break \EndIIf \Comment{Abbbruchbedingung}
    \EndFor
    \EndFunction
  \end{algorithmic}
\end{frame}

%%%%%%%%%%%%%%%%%%%%%%%%%%%%%%%%%%%%%%%%%%%%%%%%%%%%%%%%%%%%%%%%%%%%%%%%%%%%
% Numerische Ergebnisse
%%%%%%%%%%%%%%%%%%%%%%%%%%%%%%%%%%%%%%%%%%%%%%%%%%%%%%%%%%%%%%%%%%%%%%%%%%%%
\section{Numerische Ergebnisse}\label{sec:Ergebnisse}
\begin{frame}{Numerische Ergebnisse}
  \centering \textcolor{red}{\huge TODO}
  \begin{tikzpicture}\begin{loglogaxis}[width=8cm,
    xlabel={\#Gitterpunkte $N$ in einer Dimension}, ylabel={\#Iterationen}, grid=minor,
    log basis x=2, legend pos= outer north east,
    legend entries={MG, CG, RBSOR, SOR, $\mathcal{O}(N)$},
    domain=4:128]
    %\addplot table {it_MG.csv};
    %\addplot table {it_CG.csv};
    %\addplot table {it_RB.csv};
    %\addplot table {it_SOR.csv};
    \addplot[black, dashed] {8*x};
  \end{loglogaxis}\end{tikzpicture}
\end{frame}

\begin{frame}{Numerische Ergebnisse}
  \centering
  \begin{tikzpicture}\begin{loglogaxis}[width=8cm,
    xlabel={\#Gitterpunkte $N$ in einer Dimension}, ylabel={Laufzeit}, grid=minor,
    log basis x=2, legend pos= outer north east,
    legend entries={MG, CG, RBSOR, SOR, $\mathcal{O}(N^3)$, $\mathcal{O}(N^2)$},
    domain=16:256]
    %\addplot table {speedup/drivencavity_1MG.csv};
    %\addplot table {speedup/drivencavity_1CG.csv};
    %\addplot table {speedup/drivencavity_1RBSOR.csv};
    %\addplot table {speedup/drivencavity_1SOR.csv};
    \addplot[black, dashed] {x^3/1500};
    \addplot[black, dotted] {x^2/600};
  \end{loglogaxis}\end{tikzpicture}
\end{frame}

\begin{frame}{Numerische Ergebnisse}
  \centering
  \begin{tikzpicture}\begin{loglogaxis}[width=8cm,
    xlabel={\#Gitterpunkte $N$ in einer Dimension}, ylabel={Laufzeit}, grid=minor,
    log basis x=2, legend pos= outer north east,
    legend entries={MG, CG, RBSOR, SOR, $\mathcal{O}(N^3)$, $\mathcal{O}(N^2)$},
    domain=16:256]
    %\addplot table {speedup/drivencavity_2parallelMG.csv};
    %\addplot table {speedup/drivencavity_2parallelCG.csv};
    %\addplot table {speedup/drivencavity_2parallelRBSOR.csv};
    %\addplot table {speedup/drivencavity_2parallelSOR.csv};
    \addplot[black, dashed] {x^3/1500};
    \addplot[black, dotted] {x^2/600};
  \end{loglogaxis}\end{tikzpicture}
\end{frame}

\begin{frame}{Numerische Ergebnisse}
  \centering
  \begin{tikzpicture}\begin{semilogxaxis}[width=8cm,
    xlabel={\#Gitterpunkte $N$ in einer Dimension}, ylabel={Speedup},
    log basis x=2, legend pos= outer north east, grid=major,
    legend entries={MG (2), CG (2), RBSOR (2), SOR (2), MG (4), CG (4), RBSOR (4), SOR (4)}]
    \addplot table {speedup/speedup2_MG.csv};
    \addplot table {speedup/speedup2_CG.csv};
    \addplot table {speedup/speedup2_RB.csv};
    \addplot table {speedup/speedup2_SOR.csv};
    \addplot table {speedup/speedup4_MG.csv};
    \addplot table {speedup/speedup4_CG.csv};
    \addplot table {speedup/speedup4_RB.csv};
    \addplot table {speedup/speedup4_SOR.csv};
  \end{semilogxaxis}\end{tikzpicture}
\end{frame}

%%%%%%%%%%%%%%%%%%%%%%%%%%%%%%%%%%%%%%%%%%%%%%%%%%%%%%%%%%%%%%%%%%%%%%%%%%%%
% Fazit & Ausblick
%%%%%%%%%%%%%%%%%%%%%%%%%%%%%%%%%%%%%%%%%%%%%%%%%%%%%%%%%%%%%%%%%%%%%%%%%%%%
\section{Fazit und Ausblick}\label{sec:Fazit}
\begin{frame}{Fazit und Ausblick}
  Fazit:
  \begin{itemize}[<+(1)->]
    \item CG: gleiche Konvergenzgeschwindigkeit wie (RB)SOR mit optimalem
        Parameter (für komplexe Geometrie unbekannt), aber keine freien Parameter
    \item MG: bessere Konvergenzgeschwindigkeit, benötigt aber guten Glätter
        und ist problemspezifisch
  \end{itemize}
  Ausblick:
  \begin{itemize}
    \item Vorkonditionierte CG Verfahren
    \item Mehrgitter für Gitter mit Hindernissen
    \item Lösung der Gesamten Navier-Stokes-Gleichungen als System mittels MG
    \item Adaptive Mehrgitter Methode (Adaptive lokale Gitterlevel,
        hierarchische Gitter)
    \item Full Multigrid und Full Approximation Scheme (Nichtlineare Probleme)
    \item Kombination CG und MG (sogenannter Krylovbeschleunigte MG)
  \end{itemize}
\end{frame}

\begin{frame}[plain]
  \begin{center}
    \Large \textcolor{simtechred}{ Vielen Dank für die Aufmerksamkeit } \\[2em]
    \normalsize Noch Fragen?
  \end{center}
\end{frame}

\end{document}
